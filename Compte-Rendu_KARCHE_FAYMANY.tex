\documentclass[a4paper, 11pt]{article}
\usepackage[top=2cm, bottom=2cm, left=2cm, right=2cm]{geometry}
\usepackage[utf8]{inputenc}
\usepackage[OT1]{fontenc}
\usepackage[french]{babel}
\usepackage{graphicx}

\pagestyle{headings}

\title{Projet Développement Web}

\author{KARCHE Mickael, FAYMANY Tanouvil}

\date{\today}

\begin{document}

\maketitle

\newpage

\tableofcontents

\newpage

\section{Description du site}
Pour ce projet, nous avons mis en place un site qui concerne l'organisation de LAN Party - événement rassemblant des personnes dans le but de jouer à des jeux vidéos - à Saint-Etienne plus précisément.
Le site permettra entre autre d'organiser une lan et contiendra donc toutes les informations nécessaires pour son bon déroulement.


\section{Mise en place du site}
Avant de commencer la création du site, nous avons d'abord réfléchi à la structure générale du site (Qu'allions nous mettre en place, de quoi avions nous besoin pour créer un site).

\subsection{Structure du site}
Le site est divisé principalement en 5 sections.
\begin{enumerate}
\item Accueil : C'est la page d'accueil du site lorsqu'on la visite pour la 1ère fois.
\item Infos : Cette partie est divisé en plusieurs sous-parties. C'est ici que vous trouverez toutes les informations à savoir si vous voulez participer à cette événement.
\item Partenaires : Contient tout les partenaires de notre site.
\item Forum : Ce forum servira avant tout à discuter, les personnes pourront librement parler de leur passion qu'est les jeux vidéos.
  Cela peut être aussi un forum d'entraide: Si une personne est nouvelle et a besoin d'aide pour pouvoir progresser, ce forum sera le bienvenu.
  Il permettra aussi de partager des idées à propos de la LAN (Améliorations possibles, Point à corriger, etc.).
  Il peut aussi permettre la rencontre entre des personnes qui chercheraientt d'autres partenaires pour pouvoir y participer sachant que le jeu principal du tournoi de la Zanga Esport est League of Legends, un jeu d'équipe en 5 contre 5.
  %%%%%%%%%%%% A completer %%%%%%%%%%%%%%%%%%
\item Contact : Cette section pourra être utile si jamais vous ne trouvez pas de réponse à vos questions lors de votre visite sur ce site.
\end{enumerate}
Il y a aussi 2 boutons pour se connecter ou s'inscrire.
%%%%%%%%%%%% A completer %%%%%%%%%%%%%%%%%%

\section{Gestion d'utilisateur}
Il existe 2 types d'utilisateur: l'utilisateur lambda et les admins
Lorsque vous accédez au site, vous aurez la possibilité de vous inscrire ou de vous connecter.\\
Si vous vous inscrivez, il vous suffira de cliquer sur le boutton \textit{S'inscrire} en haut à droite de la page, qui vous redirigera sur une page où l'on
vous demandera les éléments suivants :
\begin{enumerate}
\item Une adresse e-mail valide
\item Un pseudo qui sera votre donc votre nom d'utilisateur, le pseudo ne doit pas être déjà existant, le cas échéant l'inscription ne pourra se faire.
\item Un mot de passe.
\item La confirmation du mot de passe, les 2 mots de passe doivent être évidemment les même.
\end{enumerate}
A la fin de votre inscription, vous recevrez un mail contenant votre nom d'utilisateur ainsi que le mot de passe que vous avez choisi.\\
Une fois inscrit, vous devrez donc pouvoir vous connecter en cliquant sur le bouton \textit{Connexion} situé en haut à droite: cela vous redirigera sur page où l'on vous demandera votre pseudo et votre mot de passe pour pouvoir vous connecter.\\
Une fois connecté, la personne connecté aura donc la possibilité d'accéder à certains contenus, ainsi que d'utiliser pleinement le forum.\\
Elle pourra créer des nouveaux sujets, répondre à des messages et elle aura aussi la possibilité de supprimer ses propres messages.\\
A contrario si un utilisateur n'est pas inscrit, il ne pourra pas répondre à des messages sur le forum ou bien même de créer des sujets mais il pourra quand même lire les messages et les sujets postés sur le forum.\\
L'utilisateur pourra aussi changer de mot de passe s'il le souhaite: il lui suffira d'accéder à son profil en cliquant sur le boutton \textit{profil} placé en haut à droite du site puis de changer son mot de passe: à la fin de cette étape, il recevra un e-mail indiquant son nom d'utilisateur et son nouveau mot de passe.\\
L'administrateur aura des droits supplémentaires, il pourra modérer tout le contenu que les utilisateurs créeront : il pourra supprimer des messages ou carrément des sujets à sa guise.

\section{Utilisation de Javascript}
%%%%%%%%%%%% A completer %%%%%%%%%%%%%%%%%%
%%%%%%%%%%%% A completer %%%%%%%%%%%%%%%%%%
%%%%%%%%%%%% A completer %%%%%%%%%%%%%%%%%%
%%%%%%%%%%%% A completer %%%%%%%%%%%%%%%%%% 

\section{Base de données}
Pour pouvoir utiliser le forum et inscrire les utilisateurs et les admins.
Nous avons créer une base de données appelée \textit{ESPORT}.
Elle contient plusieurs tables.
\begin{enumerate}
\item %%%%%%%%%%%% A completer %%%%%%%%%%%%%%%%%% 
\item %%%%%%%%%%%% A completer %%%%%%%%%%%%%%%%%%
\end{enumerate}

\end{document}
