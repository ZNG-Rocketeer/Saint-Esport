\documentclass[a4paper, 11pt]{article}
\usepackage[top=2cm, bottom=2cm, left=2cm, right=2cm]{geometry}
\usepackage[utf8]{inputenc}
\usepackage[OT1]{fontenc}
\usepackage[french]{babel}
\usepackage{graphicx}

\pagestyle{headings}

\title{Projet Développement Web}

\author{KARCHE Mickael, FAYMANY Tanouvil}

\date{\today}

\begin{document}

\maketitle

\newpage

\tableofcontents

\newpage

\section{Description du site}
Pour ce projet, nous avons mis en place un site qui concerne l'organisation de LAN Party - événement rassemblant des personnes dans le but de jouer à des jeux vidéos - à Saint-Etienne plus précisément.
Le site permettrera entre autre d'organiser une lan et contiendra donc toutes les informations nécessaires pour le bon déroulement de votre lan.

\section{Mise en place du site}
Avant de commencer la création du site, nous avons d'abord réfléchis à la structure générale du site (Qu'allions nous mettre en place, de quoi avions nous besoin pour créer un site).

\subsection{Structure du site}
Le site est divisé principalement en 5 sections.
\begin{enumerate}
\item Accueil : C'est la page d'accueil du site lorsqu'on la visite pour la 1ère fois.
\item Infos : Cette partie est divisé en plusieurs sous-parties. C'est ici que vous trouverez toutes les informations à savoir si vous voulez participer à cette événement.
\item Partenaires : Contient tout les partenaires de notre site.
\item Forum : Ce forum servira avant tout à discuter, les personnes pourront librement parler de leur passion qu'est les jeux vidéos.
  Cela peut être aussi un forum d'entraide: Si une personne est nouvelle et a besoin d'aide pour pouvoir progresser, ce forum sera le bienvenu.
  Il permettra aussi de partager des idées à propos de la LAN (Améliorations possibles, Point à corriger, etc.).
  Il peut aussi permettre la rencontre entre des personnes qui chercheraientt d'autres partenaires pour pouvoir y participer sachant que le jeu principal du tournoi de la Zanga Esport est League of Legends, un jeu d'équipe en 5 contre 5.
  %%%%%%%%%%%% A completer %%%%%%%%%%%%%%%%%%
\item Contact : Cette section pourra être utile si jamais vous ne trouvez pas de réponse à vos questions lors de votre visite sur ce site.
\end{enumerate}
Il y a aussi 2 boutons pour se connecter ou s'inscrire.

\section{Utilisation de Javascript}


\section{Base de données}
Pour pouvoir utiliser le forum et inscrire les utilisateurs et les admins.
Nous avons créer une base de données appelée \textit{ESPORT}.
Elle contient plusieurs tables.
\begin{enumerate}
\item %%%%%%%%%%%% A completer %%%%%%%%%%%%%%%%%% 
\item %%%%%%%%%%%% A completer %%%%%%%%%%%%%%%%%%
\end{enumerate}

\end{document}
